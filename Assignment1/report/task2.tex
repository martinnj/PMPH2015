\section{Task 2}

For this task I had to finish the implementation of a function for calculating
the ``Longest Satisfying Segment''. The code written for the solution can be
senn in Figure \ref{fig:t2code}. The part I finished is path of the LLS
operator. It takes information about two adjoining segments and creates
information about the segment resulting from joining the segments together.

\begin{figure}[ht]
    \begin{lstlisting}
    connect = p [lastx,firsty]
    newlss  = lssx `max` lssy `max` (if connect then lcsx+lisy else 0)
    newlis  = lisx `max` (if (connect && okx) then (tlx+lisy) else 0)
    newlcs  = lcsy `max` (if (connect && oky) then (tly+lcsx) else 0)
    newok   = if connect then okx && oky else False
    \end{lstlisting}
    \caption{The code written for the second task of the assignment.}
    \label{fig:t2code}
\end{figure}

\begin{itemize}
    \item[Line 1] Check wether the predicate $p$ holds over the edge between the
    segments, that is: If we have a list of the last element from list $x$ and
    the first element of list $y$, does $p$ still hold. The is used to determine
    how the two lists can be connected.

    \item[Line 2] Sets the length of the new longest satisfying segment in the
    joined list based on which value is larger: The LSS from list $x$, the LSS
    from list $y$ and if the \texttt{connect} value from Line 1 is ``true'' then
    the length of the concluding segment of $x$ joined with the length of the
    initial segment of $y$.

    \item[Line 3] Sets the length of the longest satisfying initial segment for
    the joined list by selecting the maximum of either: Length of the initial
    segment of list $x$ or if the \texttt{connect} and \texttt{okx} is true, it
    will take the total length of $x$ plus the length of the initial segment of
    $y$.

    \item[Line 4] Sets the length of the longest satisfying concluding segment
    for the joined list. Uses same method as above, and the length values of
    longest satisfying concluding segment for $y$ or if \texttt{connect} and
    \texttt{oky} is set, the total length of $y$ plus the length of the
    concluding segment of $x$.

    \item[Line 5] Determines wether or not the joined segment is ok by doing an
    ``and'' on the ok status of $x$ and $y$.
\end{itemize}
