\section{Task 4}

The attached program will map the function $\symbol{92} x -> (x/(x-2.3))^3$ unto
the array $[1,..,753411]$ both sequanteially and using a CUDA kernel for doing
it in parallel. Both runs are timed and the results are compared to verify that
the CPU and GPU are agreeing on the result. For all the runs I did the results
we're close enough to satisfy the above function.

The timing is done only for the calculation and not copying data to the graphics
cards or allocating memory. The verification of the results is done using the
following check $\text{abs}(cpu_t - gpu_t) < \epsilon$. All test runs was run on
one of the compute machines we we're granted access to as part of the course.

\begin{table}
    \center
    \begin{tabular}{|c|c|c|}
        \hline
        \textbf{Array Size} & \textbf{CPU Time} & \textbf{GPU Time} \\\hline
           100 &    50 &  63 \\
           200 &    90 &  65 \\
           300 &    73 &  66 \\
           400 &    82 &  65 \\
           500 &    91 & 103 \\
           600 &   130 & 107 \\
           700 &   111 & 105 \\
           800 &   120 &  80 \\
           900 &   152 &  80 \\
          1000 &   143 &  76 \\
          1100 &   144 &  77 \\
          1300 &   162 &  68 \\
          1500 &   184 & 150 \\
          2000 &   247 &  77 \\
          3000 &   360 &  75 \\
          5000 &   482 &  77 \\
         10000 &  1122 &  95 \\
         15000 &  1320 &  80 \\
         50000 &  5657 & 149 \\
        100000 &  8233 & 173 \\
        150000 & 12266 & 244 \\
        200000 & 16357 & 278 \\
        250000 & 20409 & 341 \\
        300000 & 24426 & 356 \\
        350000 & 28524 & 434 \\
        400000 & 32492 & 444 \\
        500000 & 42428 & 578 \\
        600000 & 48733 & 635 \\
        700000 & 56853 & 747 \\
        753411 & 61204 & 806 \\\hline
    \end{tabular}
    \caption{The runtimes reported by the program, measured in microseconds.}
    \label{tab:times}
\end{table}


The timing results for different array sizes are shown in Table \ref{tab:times}
and shows that the CPU generally are only faster at very small array sizes.
According to my measurements already between 100-200 elements, the GPU code runs
faster than the sequential CPU code. Furthermore the increase in compute time
rises a lot faster for the CPU code compared to the GPU version. The reason the
GPU is so much faster as the number of iterations increase is because of it's
ability to process 1024 different ``iterations'' of the sequential loop in
parallel.

Since the measurements are in microseconds on a time shared machine there was
some inaccuracies, the values in Table \ref{tab:times} is the average of running
the program 5 times after each other.
