\section{Task 3}

For this section I implemented a filter function called
\texttt{segmSpecialFilter} which takes a predicate as well as an input list of
data, and an info list with the segmentation information for the data list. The
function then orders the segments based on the predicate result for each element
and returns the segments joined into a list and a new segmentation information
list. Figure \ref{fig:t3code} shows my implementation of the function.

\begin{figure}[H]
    \begin{lstlisting}
    segmSpecialFilter :: (a->Bool) -> [Int] -> [a] -> ([Int],[a])
    segmSpecialFilter cond sizes arr =
        let n   = length arr
            cs  = map cond arr
            tfs = map (\f -> if f then 1
                             else 0) cs

            ffs = map (\f->if f then 0
                                else 1) cs

            isT = segmScanInc (+) 0 sizes tfs

            isF = segmScanInc (+) 0 sizes ffs

            acc_sizes = scanInc (+) 0 sizes

            is = map (\s -> isT !! (s - 1)) acc_sizes
            si  = segmScanInc (+) 0 sizes sizes

            offsets = zipWith (-) acc_sizes si

            inds = map (\ (c,i,o,iT,iF) -> if c then iT+o-1 else iF+i+o-1 )
                        (zip5 cs is offsets isT isF)

            tmp1 = map (\m -> iota m) sizes
            iotas = map (+1) $ reduce (++) [] tmp1

            flags = map (\(f,i,s,ri) -> if f > 0
                                        then (if ri > 0
                                              then ri
                                              else f)
                                        else (if (i-1) == ri
                                              then s-ri
                                              else 0)) (zip4 sizes iotas si is)

        in  (flags, permute inds arr)
    \end{lstlisting}
    \caption{The code written for the third task of the assignment.}
    \label{fig:t3code}
\end{figure}
