\section{Task II.2: Vector Processors}

\subsection{Part a}
Figure \ref{fig:t22acode} shows the code converted to assembler using vector
instructions. To make the code more readable, here is a map of how registers are
used:
\begin{itemize*}
    \item[\texttt{R1}] stores the $k$ variable.
    \item[\texttt{R2}] stores the location of the current slice of $x$.
    \item[\texttt{R3}] stores the location of the current slice of $y$.
    \item[\texttt{R4}] stores loop iteration limit, in our case 1023.
    \item[\texttt{V1}] stores the current slice of $x$.
    \item[\texttt{V2}] stores the current slice of $y$.
    \item[\texttt{V3}] stores the multiplication results.
    \item[\texttt{V4}] stores the stores accumulated product vector.
\end{itemize*}


\begin{figure}
    % R1  = k
    % R2  = x slice
    % R3  = y slice
    % R4  = 1023 - loop limit.
    \begin{lstlisting}[language={[x86masm]Assembler}]
LOOP:
    L.V   V1 , 0(R2),  R8  ; Load vector x from base 0(R2)
    L.V   V2 , 0(R3),  R8  ; Load vector y from base 0(R3)
    MUL.V V3 ,   V2 ,  V1  ; multiply vector elements.
    ADD.v V4 ,   V4 ,  V3  ; then add to accumulated result (v4)
    ADDI  R2 ,   R2 , #64  ; Set R2 to point to next slice
    ADDI  R3 ,   R3 , #64  ; Set R2 to point to next slice
    ADDI  R1 ,   R1 , #64  ; Increase counter.
    BNE   R1 ,   R4 , LOOP ; If k != 1023, run loop again.
    \end{lstlisting}
    \caption{The vectorized version of the code from the assignment.}
    \label{fig:t22acode}
\end{figure}

\subsection{Part b}
% Memory latency = 30
% Multiply latency = 10
% Add latency = 5

If the vector size is 1024, each register will be able to hold all of $x$ og $y$
in a single register, meaning the loop will run through once and exit.

The first line is a label and so takes no clicks, line 2-3 takes $30$ clocks
each in order to fetch the vectors. Line 4 takes $10$ clocks to execute and line
5 takes $5$ clocks. Line 6-9 all take $1$ clock each totalling $4$ clocks.
The total execution time for the entires loop iteration is $2 \cdot 30 + 10 + 5
+ 4 = 79$ clocks.


\subsection{Part c}

Since the matrix multiplication is a single dot product for each element in the
resulting matrix. In the previus subsection I calculated that a single dot
product takes $79$ clocks, for the $1024 \times 1024$ matrix the total time
(excluding any overhead for looping over the entries and allocating space etc)
will be $79 \cdot 1024 \cdot 1024 = 82837504$ clocks.
