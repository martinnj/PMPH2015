\section{Task II.2: Vector Processors}

\subsection{Part a}
Figure \ref{fig:t22acode} shows the code converted to assembler using vector
instructions. To make the code more readable, here is a map of how registers are
used:
\begin{itemize*}
    \item[\texttt{R1}] stores the $k$ variable.
    \item[\texttt{R2}] stores the address of the $x$ array.
    \item[\texttt{R3}] stores the address of the $y$ array.
    \item[\texttt{R4}] stores the address of the $x$ array offset with $k$.
    \item[\texttt{R5}] stores the address of the $y$ array offset with $k$.
    \item[\texttt{R6}] stores the address of the $p$ result array.
    \item[\texttt{R7}] stores the address of the $p$ result array offset with $k$.
    \item[\texttt{R8}] stores the stride.
    \item[\texttt{R9}] stores a dot product of the two slices until copied away.
    \item[\texttt{R10}] stores the value 1023 for use in the loop counter.
    \item[\texttt{R10}] stores an loop iteration counter.
    \item[\texttt{V1}] stores the retrieved slice of $x$.
    \item[\texttt{V2}] stores the retrieved slice of $y$.
\end{itemize*}

For the conversion of the loop I assumed that the hardware was capable of
producing the dot product of two vectors using a single instruction. I made the
assumption since I was unable to find any documentation for which instructions
we're actually available on the hardware/assembler language. The assignment text
states that we have  a multiplication pipline, however I am unsure if this is
only capable of multiplying a scalar to each element in the vector or if it can
do dot- or cross-products as well.

\begin{figure}[ht]
    % R1  = k
    % R2  = &x
    % R3  = &y
    % R4  = &x + k
    % R5  = &x + k
    % R6  = &p
    % R8  = stride
    % R9  = temp p value
    % R10 = 1023, k limit
    % R11 = iteration number.
    % R12 = &p + iteration number
    \begin{lstlisting}[language={[x86masm]Assembler}]
    ADDI  R8 , R0, #1     // Set stride to 1
    ADDI  R11, R0, #0     // Set counter = 0
    ADDI  R1 , R0, #0     // Set k = 0
    ADDI  R10, R0, #1023  // Set upper limit for k.
OUTERLOOP:
    ADD   R4 , R2, R1     // Get &x + k-offset
    ADD   R5 , R3, R1     // Get &y + k-offset
    ADD   R12, R6, R11    // Get &p + iteration-offset
INNERLOOP:
    L.V   V1 , 0(R2), R8  // Load vector x from base 0(R2)
    L.V   V2 , 0(R3), R8  // Load vector y from base 0(R3)
    DOT.V R9 , V1   , V2  // Perform the dot product.
    SW    R12, R9         // Store dotproduct of slice on &p + iteration-offset
    ADDI  R1 , R1   , #64 // k += 64
    ADDI  R11, R11  , #1  // Increase counter.
CHECK:
    BNE   R1, R10, OUTERLOOP // If k != 1023, run loop again.
    \end{lstlisting}
    \caption{The vectorized version of the code from the assignment.}
    \label{fig:t22acode}
\end{figure}

\subsection{Part b}
% Memory latency = 30
% Multiply latency = 10
% Add latency = 5

If the vector size is 1024, each register will be able to hold all of $x$ og $y$
in a single register, meaning the loop will run through once and exit.

The first line will take $5$ clocks to start due to the initial latency, lines
2-4 will take cost $1$ clock each. The first 3 lines will then take
$5+1+1+1 = 7$ clocks to complete.

Lines $6-8$ additionally takes $1$ clock each as the pipeline is already
startet, meaning the cost is $1+1+1 = 3$.

Line 10-11 takes $30$ cycles to complete each to access the vector bank. The
multiplication/dot-product have a latency of $10$ clocks and the storage back
to memory on line 13 takes an additional $30$ clocks. The two increases on
line 14-15 takes $1$ clock each. The total for lines 10-15 is then
$30+30+10+30+1+1 = 102$ clocks.

The last branch takes $1$ clock to execute, so the total execution time for the
entires loop iteration is $7 + 3 + 102 + 1 = 113$ clocks.


\subsection{Part c}

Since the matrix multiplication is a single dot product for each element in the
resulting matrix. In the previus subsection I calculated that a single dot
product takes $113$ clocks, for the $1024 \text{ x } 1024$ matrix the total time
(excluding any overhead for looping over the entries and allocating space etc)
will be $113 \cdot 1024 \cdot 1024 = 118489088$ clocks.
