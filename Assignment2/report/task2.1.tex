\section{Task II.1: Pipelining}

Assuming a 5-stage pipeline design as described in \cite{l3InOrder} and given
the program in Figure \ref{fig:t21code} 5 questions (a-e) need to be answered.

\begin{figure}
    \begin{lstlisting}[language={[x86masm]Assembler}]
    SEARCH:     LW R5, 0(R3)        /I1 Load item
                SUB R6, R5, R2      /I2 compare with key
                BNEZ R6, NOMATCH    /I3 check for match
                ADDI R1, R1, #1     /I4 count matches
    NOMATCH:    ADDI R3, R3, #4     /I5 next item
                BNE R4, R3, SEARCH  /I6 continue until all items
    \end{lstlisting}
    \caption{Code given in the assignment text.}
    \label{fig:t21code}
\end{figure}

\subsection{a) Inserting NOOPS}

\begin{figure}
    \begin{lstlisting}[language={[x86masm]Assembler}]
    SEARCH:     LW R5, 0(R3)        /I1 Load item
                NOP
                NOP
                NOP
                SUB R6, R5, R2      /I2 compare with key
                NOP
                NOP
                NOP
                BNEZ R6, NOMATCH    /I3 check for match
                NOP
                NOP
                ADDI R1, R1, #1     /I4 count matches
    NOMATCH:    ADDI R3, R3, #4     /I5 next item
                NOP
                NOP
                NOP
                BNE R4, R3, SEARCH  /I6 continue until all items
                NOP
                NOP
    \end{lstlisting}
    \caption{Code after insertion of \texttt{NOOP}s.}
    \label{fig:t21a}
\end{figure}

\subsection{b) Execution time (clocks) with stalling}
\begin{itemize}
    \item[Hit: ] Table \ref{tab:t21bhit} shows  the execution of the code, the
    last execution step finishes at clock 14.

    \item[No Hit: ] Table \ref{tab:t21bmiss} shows  the execution of the code, the
    last execution step finishes at clock 17.
\end{itemize}

\subsection{c) Execution time (clocks) with stalling and register forwarding}
\begin{itemize}
    \item[Hit: ] Table \ref{tab:t21chit} shows  the execution of the code, the
    last execution step finishes at clock 11.

    \item[No Hit: ] Table \ref{tab:t21cmiss} shows  the execution of the code, the
    last execution step finishes at clock 12.
\end{itemize}

\subsection{d) Execution time (clocks) with stalling and full forwarding}
\begin{itemize}
    \item[Hit: ] Table \ref{tab:t21dhit} shows  the execution of the code, the
    last execution step finishes at clock 9.

    \item[No Hit: ] Table \ref{tab:t21dmiss} shows  the execution of the code, the
    last execution step finishes at clock 10.
\end{itemize}




\begin{landscape}
    \begin{table}[]
    \centering
    \begin{tabular}{llllllllllllllll}
    \hline
             &                                         & c1 & c2 & c3 & c4 & c5 & c6 & c7 & c8 & c9 & c10 & c11 & c12 & c13 & c14 \\ \hline
    SEARCH:  & \multicolumn{1}{l|}{LW R5, 0(R3)}       & IF & ID & EX & ME & WB &    &    &    &    &     &     &     &     &     \\
             & \multicolumn{1}{l|}{SUB R6, R5, R2}     &    & IF & ID & ID & ID & EX & ME & WB &    &     &     &     &     &     \\
             & \multicolumn{1}{l|}{BNEZ R6, NOMATCH}   &    &    & IF & IF & IF & ID & ID & ID & EX &     &     &     &     &     \\
             & \multicolumn{1}{l|}{ADDI R1, R1, \#1}   &    &    &    &    &    & IF & IF & IF & ID & EX  & ME  & WB  &     &     \\
    NOMATCH: & \multicolumn{1}{l|}{ADDI R3, R3, \#4}   &    &    &    &    &    &    &    &    & IF & ID  & EX  & ME  & WB  &     \\
             & \multicolumn{1}{l|}{BNE R4, R3, SEARCH} &    &    &    &    &    &    &    &    &    & IF  & ID  & ID  & ID  & EX  \\ \hline
    \end{tabular}
    \caption{Execution table showing where in the pipeline different instructions are for different clocks if there is a hit in the search loop.}
    \label{tab:t21bhit}
    \end{table}

    \begin{table}[]
    \centering
    \begin{tabular}{lllllllllllllllllll}
    \hline
             &                                         & c1 & c2 & c3 & c4 & c5 & c6 & c7 & c8 & c9                      & c10 & c11 & c12 & c13 & c14 & c15 & c16 & c17 \\ \hline
    SEARCH:  & \multicolumn{1}{l|}{LW R5, 0(R3)}       & IF & ID & EX & ME & WB &    &    &    &                         &     &     &     &     &     &     &     &    \\
             & \multicolumn{1}{l|}{SUB R6, R5, R2}     &    & IF & ID & ID & ID & EX & ME & WB &                         &     &     &     &     &     &     &     &     \\
             & \multicolumn{1}{l|}{BNEZ R6, NOMATCH}   &    &    & IF & IF & IF & ID & ID & ID & \multicolumn{1}{l|}{EX} &     &     &     &     &     &     &     &     \\
             & \multicolumn{1}{l|}{ADDI R1, R1, \#1}   &    &    &    &    &    &    &    & IF & \multicolumn{1}{l|}{ID} &     &     &     &     &     &     &     &     \\
    NOMATCH: & \multicolumn{1}{l|}{ADDI R3, R3, \#4}   &    &    &    &    &    &    &    &    & \multicolumn{1}{l|}{IF} & IF  & ID  & EX  & ME  & WB  &     &     &     \\
             & \multicolumn{1}{l|}{BNE R4, R3, SEARCH} &    &    &    &    &    &    &    &    &                         &     & IF  & ID  & ID  & ID  & EX  & ME  & WB  \\ \hline
    \end{tabular}
    \caption{Execution table showing where in the pipeline different instructions are for different clocks if there is no hit in the search loop. The vertical line in the middle of the table indicates that \texttt{IF} and \texttt{ID} was flushed.}
    \label{tab:t21bmiss}
    \end{table}
\end{landscape}

\begin{landscape}

    \begin{table}[]
    \centering
    \begin{tabular}{llllllllllllllll}
    \hline
             &                                         & c1 & c2 & c3 & c4 & c5 & c6 & c7 & c8 & c9 & c10 & c11 \\ \hline
    SEARCH:  & \multicolumn{1}{l|}{LW R5, 0(R3)}       & IF & ID & EX & ME & WB &    &    &    &    &     &     \\
             & \multicolumn{1}{l|}{SUB R6, R5, R2}     &    & IF & ID & ID & ID & EX & ME & WB &    &     &     \\
             & \multicolumn{1}{l|}{BNEZ R6, NOMATCH}   &    &    & IF & IF & IF & ID & EX &    &    &     &     \\
             & \multicolumn{1}{l|}{ADDI R1, R1, \#1}   &    &    &    &    &    & IF & ID & EX & ME & WB  &     \\
    NOMATCH: & \multicolumn{1}{l|}{ADDI R3, R3, \#4}   &    &    &    &    &    &    & IF & ID & EX & ME  & WB  \\
             & \multicolumn{1}{l|}{BNE R4, R3, SEARCH} &    &    &    &    &    &    &    & IF & ID & EX  &     \\ \hline
    \end{tabular}
    \caption{Execution table when using stalling and register forwarding. This is the result of the loop does hit a match}
    \label{tab:t21chit}
    \end{table}

    \begin{table}[]
    \centering
    \begin{tabular}{llllllllllllllll}
    \hline
             &                                         & c1 & c2 & c3 & c4 & c5 & c6 & c7                      & c8 & c9 & c10 & c11 & c12 \\ \hline
    SEARCH:  & \multicolumn{1}{l|}{LW R5, 0(R3)}       & IF & ID & EX & ME & WB &    &                         &    &    &     &     &     \\
             & \multicolumn{1}{l|}{SUB R6, R5, R2}     &    & IF & ID & ID & ID & EX & ME                      & WB &    &     &     &     \\
             & \multicolumn{1}{l|}{BNEZ R6, NOMATCH}   &    &    & IF & IF & IF & ID & EX                      &    &    &     &     &     \\
             & \multicolumn{1}{l|}{ADDI R1, R1, \#1}   &    &    &    &    &    & IF & \multicolumn{1}{l|}{ID} &    &    &     &     &     \\
    NOMATCH: & \multicolumn{1}{l|}{ADDI R3, R3, \#4}   &    &    &    &    &    &    & \multicolumn{1}{l|}{IF} & IF & ID & EX  & ME  & WB  \\
             & \multicolumn{1}{l|}{BNE R4, R3, SEARCH} &    &    &    &    &    &    &                         &    & IF & ID  & EX  &     \\ \hline
    \end{tabular}
    \caption{Execution table when using stalling and register forwarding. This is the result of the loop does not hit a match. The vertical line in the middle of the table indicates that \texttt{IF} and \texttt{ID} was flushed.}
    \label{tab:t21cmiss}
    \end{table}

\end{landscape}

\begin{landscape}
    \begin{table}[]
    \centering
    \begin{tabular}{lllllllllll}
    \hline
             &                                         & c1 & c2 & c3 & c4 & c5 & c6 & c7 & c8 & c9 \\ \hline
    SEARCH:  & \multicolumn{1}{l|}{LW R5, 0(R3)}       & IF & ID & EX & ME & WB &    &    &    &    \\
             & \multicolumn{1}{l|}{SUB R6, R5, R2}     &    & IF & ID & EX & ME & WB &    &    &    \\
             & \multicolumn{1}{l|}{BNEZ R6, NOMATCH}   &    &    & IF & ID & EX &    &    &    &    \\
             & \multicolumn{1}{l|}{ADDI R1, R1, \#1}   &    &    &    & IF & ID & EX & ME & WB &    \\
    NOMATCH: & \multicolumn{1}{l|}{ADDI R3, R3, \#4}   &    &    &    &    & IF & ID & EX & ME & WB \\
             & \multicolumn{1}{l|}{BNE R4, R3, SEARCH} &    &    &    &    &    & IF & ID & EX &    \\ \hline
    \end{tabular}
    \caption{Execution table when using stalling and full forwarding. This is the result of the loop does hit a match.}
    \label{tab:t21dhit}
    \end{table}

    \begin{table}[]
    \centering
    \begin{tabular}{llllllllllllllll}
    \hline
             &                                         & c1 & c2 & c3 & c4 & c5                      & c6 & c7 & c8 & c9 & c10 \\ \hline
    SEARCH:  & \multicolumn{1}{l|}{LW R5, 0(R3)}       & IF & ID & EX & ME & WB                      &    &    &    &    &     \\
             & \multicolumn{1}{l|}{SUB R6, R5, R2}     &    & IF & ID & EX & ME                      & WB &    &    &    &     \\
             & \multicolumn{1}{l|}{BNEZ R6, NOMATCH}   &    &    & IF & ID & EX                      &    &    &    &    &     \\
             & \multicolumn{1}{l|}{ADDI R1, R1, \#1}   &    &    &    & IF & \multicolumn{1}{l|}{ID} &    &    &    &    &     \\
    NOMATCH: & \multicolumn{1}{l|}{ADDI R3, R3, \#4}   &    &    &    &    & \multicolumn{1}{l|}{IF} & IF & ID & EX & ME & WB  \\
             & \multicolumn{1}{l|}{BNE R4, R3, SEARCH} &    &    &    &    &                         &    & IF & ID & EX &     \\ \hline
    \end{tabular}
    \caption{Execution table when using stalling and full forwarding. This is the result of the loop does not hit a match. The vertical line in the middle of the table indicates that \texttt{IF} and \texttt{ID} was flushed.}
    \label{tab:t21dmiss}
    \end{table}
\end{landscape}



\subsection{e) Loop unrolling}
\todo[inline]{Will loop unrolling help?}
