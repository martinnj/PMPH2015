\section{Task 2}

For this task we are given the following code:
\begin{figure}
    \begin{lstlisting}
for i from 0 to N-1 // outer loop
    accum = A[i,0] * A[i,0];
    B[i,0] = accum;
    for j from 1 to 63 // inner loop
        tmpA = A[i, j];
        accum = sqrt(accum) + tmpA*tmpA;
        B[i,j] = accum;
    \end{lstlisting}
    \caption{Pseudocode given for the task.}
    \label{fig:t2code1}
\end{figure}

\subsection{Task 2.a}

\begin{itemize}
    \item Because both \texttt{accum} and \texttt{tmpA} are declared outside the
    loop, the outer loop cannot be parallel since the different threads will
    attempt to write to the same memory locations.

    \begin{figure}
        \begin{lstlisting}
for i from 0 to N-1 // outer loop
    float accum = A[i,0] * A[i,0];
    B[i,0] = accum;
    for j from 1 to 63 // inner loop
        float tmpA = A[i, j];
        accum = sqrt(accum) + tmpA*tmpA;
        B[i,j] = accum;
        \end{lstlisting}
        \caption{Code rewritten for allow for parallelisation of the outer
            loop.}
        \label{fig:t2code2}
    \end{figure}

    \item Since \texttt{accum} is declared outside the inner loop, it is not
    parallelisable since different threads would all try to access the same
    instance of \texttt{accum}.


\end{itemize}


\subsection{Task 2.c}
\subsection{Task 2.d}
