\section{Task 6}
% Given a n-by-n tori and an n-dimensional hypercube decide things for node
% count 4, 16, 64, 256

\subsection{Task 6.a}
% At what scale does the hypercube provide (strictly) higher bisection width than the torus?

Table \ref{tab:t6a} shows the bisection width of each network at the given
sizes, the width of the hyper-cube have a strictly greater width at $n=16$ and
higher.

\begin{table}[]
\centering
\begin{tabular}{l|ll}
        & Tori  & HyperCube                   \\ \hline
Formula & $2n$  & $2^{n-1}$                   \\
$n=4$   & $8$   & $8$                         \\
$n=16$  & $32$  & $32768$                     \\
$n=64$  & $128$ & $9.223372037\times 10^{18}$ \\
$n=256$ & $516$ & $5.789604462\times 10^{76}$
\end{tabular}
\caption{The bisection width of the two networks at given sizes.}
\label{tab:t6a}
\end{table}

\subsection{Task 6.b}
% Determine the network diameter and the switch degree for the scale at which the hypercube
% provides a higher bisection width than the torus.

% Scale: n=16.
% Network Diameter: 16 for both.
% Switch degree: 4 for tori 16 for hyper-cube.

Using the formulas from \cite[slide 38]{l7Interconnect} for scale $n=16$, the
network diameter is $16$ for both the tori and the hyper-cube. The switch degree
is 4 for the tori and 16 for the hyper-cube.

\subsection{Task 6.c}
% What can you say about the relative merits of the two topologies?
\begin{itemize}
    \item[Hyper-cybe] The high bisection width allows for potential high
    bandwidth through the many links, however this also comes with a high
    switching degree, and for higher dimensions hyper-cubes, they layout of the
    cube itself becomes highly complicated, which makes the hyper-cube a very
    costly interconnection pattern.

    \item[Torus] A constant switching degree means that nodes can be added
    without adding complexity to the rest of the system. The slower growing
    bisection width means that potential bandwidth doesn't increase as fast as
    with the hyper-cube.
\end{itemize}
