\section{Task 6}
% Given a n-by-n tori and an n-dimensional hypercube decide things for node
% count 4, 16, 64, 256

% N   | tori n | cube k
% 4       2        2
% 16      4        4
% 64      8        6
% 256    16        8
\subsection{Task 6.a}
% At what scale does the hypercube provide (strictly) higher bisection width than the torus?

Table \ref{tab:task6a1} shows the different dimensions of the two patterns for
the specified networks sizes, this is used to calculate the bisection widths
which can be seen in Table \ref{tab:task6a2}. We can see that at $N=64$ the
hypercube provides a strictly greater bisection width.

\begin{table}[H]
    \centering
    \begin{tabular}{|r|r|r|}
        \hline
        $N$ & Tori $n$ & Cube $k$ \\\hline
          4 &  2       & 2        \\
         16 &  4       & 4        \\
         64 &  8       & 6        \\
        256 & 16       & 8        \\\hline
    \end{tabular}
    \caption{The different variables needed for calculating the dimensions of
        the two patterns.}
    \label{tab:task6a1}
\end{table}

\begin{table}[H]
    \centering
    \begin{tabular}{|l|l|l|}
        \hline
        $N$     & Tori & HyperCube \\ \hline
        Formula & $2n$ & $2^{k-1}$ \\
        4       &  4   &   2       \\
        16      &  8   &   8       \\
        64      & 16   &  32       \\
        256     & 32   & 128       \\ \hline
    \end{tabular}
    \caption{My caption}
    \label{tab:task6a2}
\end{table}


\subsection{Task 6.b}
% Determine the network diameter and the switch degree for the scale at which the hypercube
% provides a higher bisection width than the torus.

% Scale: N=64.
% n/k = 8/6
% Network Diameter: 8/6
% Switch degree: 4/6

Using the formulas from \cite[slide 38]{l7Interconnect} for scale $N=64$ as
calculated in the previus sub task, we have a $8$-by-$8$ tori and a
$6$-dimensional hypercube, so we have $n=8$ and $k=6$. This gives the tori a
network diameter of $8$ and the hypercube a network diameter of $6$. The switch
degree for the tori is $4$ regardless of the size of the tori, and for the
hypercube it is $6$.

\subsection{Task 6.c}
% What can you say about the relative merits of the two topologies?
\begin{itemize}
    \item[Hyper-cybe] The high bisection width allows for potential high
    bandwidth through the many links, however this also comes with a high
    switching degree, and for higher dimensions hyper-cubes, they layout of the
    cube itself becomes highly complicated, which makes the hyper-cube a very
    costly interconnection pattern.

    \item[Torus] A constant switching degree means that nodes can be added
    without adding complexity to the rest of the system. The slower growing
    bisection width means that potential bandwidth doesn't increase as fast as
    with the hyper-cube.
\end{itemize}
