\documentclass[a4paper,11pt]{article}
\usepackage{a4wide}
\usepackage[utf8x]{inputenc}
\usepackage{ucs}
\usepackage[T1]{fontenc}
\linespread{1.2}
\usepackage{amsmath,amssymb,amsthm,amsfonts,ulem}
\usepackage{courier}
% \usepackage{fourier}
\usepackage{color}
% \usepackage{clrscode3e}
\usepackage{multicol}
%\usepackage{pdflscape}
\setcounter{secnumdepth}{2}
\setcounter{tocdepth}{3}

\usepackage{hyperref}
\usepackage{listings}
\usepackage{subcaption}
\usepackage{float}
\usepackage{mdwlist}
\usepackage{wrapfig}
\usepackage{caption}
\usepackage{todonotes}
\usepackage{ulem}

\usepackage{pdflscape}

\usepackage{mathtools}
\DeclarePairedDelimiter{\ceil}{\lceil}{\rceil}

% usage: \graphicc{width}{file}{caption}{label}
\newcommand{\graphicc}[4]{\begin{figure}[H] \centering
            \includegraphics[width={#1\textwidth}, keepaspectratio=true]{{#2}}
            \caption{{#3}} \label{#4} \end{figure}}

% usage: \codefig{label}{file}{firstline}{lastline}{description}
\newcommand{\codefig}[5]
{
\begin{figure}[H]
    \lstinputlisting[firstnumber=#3,firstline=#3,lastline=#4]{#2}
    \caption{#5 (#2)}
    \label{code:#1}
\end{figure}
}

\definecolor{comment}{rgb}      {0.38, 0.62, 0.38}
\definecolor{keyword}{rgb}      {0.10, 0.10, 0.81}
\definecolor{identifier}{rgb}   {0.00, 0.00, 0.00}
\definecolor{string}{rgb}       {0.50, 0.50, 0.50}

\lstset
{
    language=c++,
    % general settings
    numbers=left,
    frame=single,
    basicstyle=\footnotesize\ttfamily,
    tabsize=2,
    breaklines=true,
    showstringspaces=false,
    % syntax highlighting
    commentstyle=\color{comment},
    keywordstyle=\color{keyword},
    identifierstyle=\color{identifier},
    stringstyle=\color{string},
}

\title{\textbf{Assignment 4} \\ Programming Massively Parallel Hardware 2015}
\author
{
    Martin Jørgensen \\
    University of Copenhagen \\
    Department of Computer Science \\
    {\tt tzk173@alumni.ku.dk}
}
\date{\today}

\begin{document}

\maketitle

\tableofcontents
\pagebreak

%%%%%%%%%%% Content sections begins here %%%%%%%%%%%

%\input{intro}
\section{Task 1: Matrix Transpose}
The code for this assignment can be found in the \textit{src/Task1} folder
handed in with this report. All functions (where applicable) have been templated
so the datatype contianed in the matrices can be any class \texttt{T}.

The code ships with a make file that supports the \texttt{clean},
\texttt{compile} and \texttt{run} targets. The current setup of the file will
run both GPU kernels on an assymetric matrix and compare to the CPU runs.

\subsection{Task 1.a: Sequential CPU Code}
The code for this subtask is implemented in the file \textit{cpuFunc.cu.h},
there is not a whole lot of spectacular work going on, the file just contains
convenience functions for creating, comparing and freeing matrices of specific
sizes. The transpose function itself is implemented in
\texttt{flatMatrixTranspose}.

\subsection{Task 1.c: Naive GPU Code}
The code for this subtask is implemented in \textit{gpuFunc.cu.h} as well as
in \textit{task1.cu}. The following functions are implemented to solve this
task:
\begin{itemize}
    \item \textit{gpuFunc.cu.h:}\begin{itemize*}
        \item \texttt{flatNaiveTransposeKernel} The actual transpose kernel.
        \item \texttt{flatMatrixCudaMalloc} Allocates device memory for a
        matrix of a given size.
        \item \texttt{flatmatrixCudaFree} Frees the device memory.
        \item \texttt{flatMatrixHostToDevice} Copies a matrix from host- to
        device memory.
        \item \texttt{flatMatrixDeviceToHost} Copies a matrix from device- to
        host memory.
    \end{itemize*}

    \item \textit{task1.cu}\begin{itemize*}
        \item \texttt{gpuNaiveTranspose} Does the setup for a transepose and
        calls the actual transpose kernel.
        \item \texttt{naiveTransposeTest} Compares the results of the CPU
        transpose and the GPU transpose as well as printing the result and how
        long each call took on average.
    \end{itemize*}
\end{itemize}

In the \texttt{main} part of the \textit{task1.cu} file, there is a region that
contains tests I used to find where the CPU transpose starts being slower than
the GPU transpose. The timings are listed in Table \ref{tab:task1time}.


\subsection{Task 1.d: Shared-Memory GPU Code}
For this task additional code is implemented in \textit{task1.cu} and
\textit{gpuFunc.cu.h}, the following methods we're implemented:
\begin{itemize}
    \item \textit{gpuFunc.cu.h:}\begin{itemize*}
        \item \texttt{flatSharedTransposeKernel} The actual transpose kernel.
    \end{itemize*}

    \item \textit{task1.cu}\begin{itemize*}
        \item \texttt{gpuSharedTranspose} Does the setup for a transepose and
        calls the actual transpose kernel.
        \item \texttt{sharedTransposeTest} Compares the results of the CPU
        transpose and the GPU shared memory kernel and prints results and
        running times.
    \end{itemize*}
\end{itemize}

\subsection{Execution Time}

Table \ref{tab:task1time} shows the running times of the CPU code and the
Shared-Memory GPU kernel. Time time is done for 10 runs of each method and the
time represents the average time for a single run. The times only cover the time
spent in the calulation kernels, and does not cover the overhead for copying
data to device memory.

\begin{table}
    \begin{tabular}{|r|r|r|r|}
        \hline
        \textbf{Matrix Dimensions} & \textbf{CPU Time} & \textbf{Naive GPU Time} & \textbf{Tiled-Memory GPU Time}\\\hline
        $100 \times  50$  & $28 \mu s$  & $31 \mu s$ & $21 \mu s$ \\
        $50 \times  100$  & $18 \mu s$  & $19 \mu s$ & $17 \mu s$ \\
        $10 \times  10$   & $>0 \mu s$  & $16 \mu s$ & $15 \mu s$ \\
        $100 \times  100$ & $37 \mu s$  & $19 \mu s$ & $17 \mu s$ \\
        $200 \times  200$ & $163 \mu s$ & $33 \mu s$ & $26 \mu s$ \\
        $300 \times  300$ & $349 \mu s$ & $57 \mu s$ & $46 \mu s$ \\
        $1024\times 1025$ & $18303 ms$  & $272\mu s$ & $145\mu s$ \\\hline

    \end{tabular}
    \caption{The running times for the CPU and GPU code as the average over 10
    runs.}
    \label{tab:task1time}
\end{table}

Table \ref{tab:task1time} shows that the CPU based code while very fast for the
first small entry is quickly overtaken by both GPU kernels. Among the GPU
kernels the Shared-Memory is the fastest from the beginning.

\section{Task 2}

For this task we are given the following code:
\begin{figure}
    \begin{lstlisting}
for i from 0 to N-1 // outer loop
    accum = A[i,0] * A[i,0];
    B[i,0] = accum;
    for j from 1 to 63 // inner loop
        tmpA = A[i, j];
        accum = sqrt(accum) + tmpA*tmpA;
        B[i,j] = accum;
    \end{lstlisting}
    \caption{Pseudocode given for the task.}
    \label{fig:t2code1}
\end{figure}

\subsection{Task 2.a}

\begin{itemize}
    \item Because both \texttt{accum} and \texttt{tmpA} are declared outside the
    loop, the outer loop cannot be parallel since the different threads will
    attempt to write to the same memory locations.

    \begin{figure}
        \begin{lstlisting}
for i from 0 to N-1 // outer loop
    float accum = A[i,0] * A[i,0];
    B[i,0] = accum;
    for j from 1 to 63 // inner loop
        float tmpA = A[i, j];
        accum = sqrt(accum) + tmpA*tmpA;
        B[i,j] = accum;
        \end{lstlisting}
        \caption{Code rewritten for allow for parallelisation of the outer
            loop.}
        \label{fig:t2code2}
    \end{figure}

    \item Since \texttt{accum} is declared outside the inner loop, it is not
    parallelisable since different threads would all try to access the same
    instance of \texttt{accum}.


\end{itemize}

\subsection{Task 2.b OpenMP Code}
The code for this can be found in \textit{src/task2openMP.cpp} and
\textit{src/cpuFunc.cu.h}. It can be built from the makefile using the
texttt{make bonus} command/target. It can then be run by running the
\textit{task2openMP} file. Execution times for different matrix sizes can be
seen in Table \ref{tab:task2times} along with execution times for the following
tasks.

\subsection{Task 2.c Naive GPU Code}
The code for this task can be found in \textit{src/task2.cu},
\textit{src/cpuFunc.cu.h}, \textit{src/gpuFunc.cu.h}. The kernel that is used
for this task is \texttt{flatNaiveTask2Kernel}. Execution times for different
matrix sizes can be seen in Table \ref{tab:task2times} along with execution
times for the following tasks.


\subsection{Task 2.d Transposed GPU Code}
The code for this task can be found in \textit{src/task2.cu},
\textit{src/cpuFunc.cu.h}, \textit{src/gpuFunc.cu.h}. The kernel that is used
for this task is \texttt{flatTransposedTask2Kernel}. It also uses the
\texttt{flatSharedTransposeKernel} from Task 1. Execution times for different
matrix sizes can be seen in Table \ref{tab:task2times} along with execution
times for the following tasks.



\subsection{Execution Time}
All running times were taken over 100 repetitions of each matrix size for each
different method. All the times we're taken running on one of the servers we
were given access to as part of the course.

\begin{table}[H]
\centering
\begin{tabular}{|r|r|r|r|r|}
\hline
\textbf{Dimensions} & \textbf{CPU} & \textbf{CPU OpenMP} & \textbf{GPU Naive} & \textbf{GPU Transposed} \\ \hline
$10 \times 64$      & $12\mu s$    & $185\mu s$          & $51\mu s$          & $69\mu s$               \\
$20 \times 64$      & $23\mu s$    & $11\mu s$           & $67\mu s$          & $68\mu s$               \\
$30 \times 64$      & $34\mu s$    & $11\mu s$           & $85\mu s$          & $70\mu s$               \\
$40 \times 64$      & $45\mu s$    & $12\mu s$           & $88\mu s$          & $68\mu s$               \\
$50 \times 64$      & $57\mu s$    & $12\mu s$           & $87\mu s$          & $70\mu s$               \\
$60 \times 64$      & $68\mu s$    & $12\mu s$           & $89\mu s$          & $70\mu s$               \\
$70 \times 64$      & $80\mu s$    & $13\mu s$           & $88\mu s$          & $74\mu s$               \\
$80 \times 64$      & $91\mu s$    & $13\mu s$           & $89\mu s$          & $69\mu s$               \\
$90 \times 64$      & $102\mu s$   & $13\mu s$           & $88\mu s$          & $71\mu s$               \\
$100 \times 64$     & $114\mu s$   & $14\mu s$           & $88\mu s$          & $74\mu s$               \\
$2000 \times 64$    & $2277\mu s$  & $106\mu s$          & $166\mu s$         & $129\mu s$              \\
$3000 \times 64$    & $3402\mu s$  & $135\mu s$          & $188\mu s$         & $166\mu s$              \\
$4000 \times 64$    & $4555\mu s$  & $179\mu s$          & $210\mu s$         & $200\mu s$              \\
$5000 \times 64$    & $5687\mu s$  & $222\mu s$          & $143\mu s$         & $145\mu s$              \\
$10000 \times 64$   & $11373\mu s$ & $436\mu s$          & $382\mu s$         & $216\mu s$              \\ \hline
\end{tabular}
\caption{This table shows the execution times for different implementations of
    the algorithm in this task. Each time is the average over 100 runs.}
\label{tab:task2times}
\end{table}

The first reading for the openMP solution is off, I have tried to re do the
reading several times and also increase the number of repetitions in order to
try and eliminate delays caused by a cold cache, but the reading is persistent.
This leads me to conclude that the overhead for creating the openMP threads are
much higher than the time saved by parallelising for a matrix with only 10 rows.

\section{Task 3}
\subsection{Task 3.a}
% Classify the misses with respect to cold, replacement, true sharing, and false sharing misses.
% Cold = Block was never referenced before
% Replacement = Block was evicted earlier because the cache was full or because another block mapped to that cache line.
% True sharing = Two procesors access same word in a block, one gets a miss/invalidated from other.
% False sharing = Tw oprocessors access different words in same block, one still gets invalidated from other.

% Example at slide 58 in L6.

The values $A$, $B$ and $C$ recide in block $B1$ and $D$ reside in block $B2$,
both blocks map to the same cache line. Table \ref{tab:t3a} shows the different
operations and their respective misses.

\begin{table}[]
\centering
\begin{tabular}{lllll}
Time                   & P1    & P2    & P3                                    & Miss Type     \\ \hline
\multicolumn{1}{l|}{1} & $R_A$ &       & \multicolumn{1}{l|}{}                 & Cold          \\
\multicolumn{1}{l|}{2} &       & $R_B$ & \multicolumn{1}{l|}{}                 & Cold          \\
\multicolumn{1}{l|}{3} &       &       & \multicolumn{1}{l|}{$R_C$}            & Cold          \\
\multicolumn{1}{l|}{4} & $W_A$ &       & \multicolumn{1}{l|}{}                 &               \\
\multicolumn{1}{l|}{5} &       &       & \multicolumn{1}{l|}{$R_D$ (evict B1)} & Cold          \\
\multicolumn{1}{l|}{6} &       & $R_B$ & \multicolumn{1}{l|}{}                 & False-Sharing \\
\multicolumn{1}{l|}{7} & $W_B$ &       & \multicolumn{1}{l|}{}                 & True-Sharing  \\
\multicolumn{1}{l|}{8} &       &       & \multicolumn{1}{l|}{$R_C$ (evict B2)} & Replacement   \\
\multicolumn{1}{l|}{9} &       & $R_B$ & \multicolumn{1}{l|}{}                 & True-Sharing
\end{tabular}
\caption{The different operations and their respective cache misses.}
\label{tab:t3a}
\end{table}

\subsection{Task 3.b}
% Which of the misses could be ignored and still guarantee that the execution is correct?
The False-Sharing miss at time $6$ could be ignored and the execution would
still be able to continue without problems.

\section{Task 4}
\todo[inline]{If time.}
% \subsection{Task 4.a}
% \subsection{Task 4.b}
% \subsection{Task 4.c}
% \subsection{Task 4.d}
% \subsection{Task 4.e}

\section{Task 5}
% We Consider a 16-by-16 torus (tori) interconnection network and determine the following
% interconnection network properties:
Given a n-by-n tori where $n=16$, we determine the requested properties.

\subsection{Task 5.a}
% network diameter
The network diameter is $16$.


\subsection{Task 5.b}
% bisection bandwidth, assuming that each link has a bandwidth of 100 Mbits/s
Bisection width is $2n = 32$, this is multiplied by the link bandwidth giving a
bisection bandwidth of $32 \cdot 100\text{Mb/s} = 3200\text{Mb/s}$.

\subsection{Task 5.c}
% the bandwidth per node
The tori have switch degree 4, so each node have 4 links with each
$100\text{Mb/s}$ this bandwidth needs to be split in both directions so the
bandwith available for each node in each link is $50\text{Mb/s}$ giving a total
bandwidth per node of
\[
\text{switch degree} \cdot \frac{bandwidth}{2} = 4 \cdot
\frac{100}{2} = 200\text{Mb/s}.
\]

\section{Task 6}
% Given a n-by-n tori and an n-dimensional hypercube decide things for node
% count 4, 16, 64, 256

\subsection{Task 6.a}
% At what scale does the hypercube provide (strictly) higher bisection width than the torus?

Table \ref{tab:t6a} shows the bisection width of each network at the given
sizes, the width of the hyper-cube have a strictly greater width at $n=16$ and
higher.

\begin{table}[]
\centering
\begin{tabular}{l|ll}
        & Tori  & HyperCube                   \\ \hline
Formula & $2n$  & $2^{n-1}$                   \\
$n=4$   & $8$   & $8$                         \\
$n=16$  & $32$  & $32768$                     \\
$n=64$  & $128$ & $9.223372037\times 10^{18}$ \\
$n=256$ & $516$ & $5.789604462\times 10^{76}$
\end{tabular}
\caption{The bisection width of the two networks at given sizes.}
\label{tab:t6a}
\end{table}

\subsection{Task 6.b}
% Determine the network diameter and the switch degree for the scale at which the hypercube
% provides a higher bisection width than the torus.

% Scale: n=16.
% Network Diameter: 16 for both.
% Switch degree: 4 for tori 16 for hyper-cube.

Using the formulas from \cite[slide 38]{l7Interconnect} for scale $n=16$, the
network diameter is $16$ for both the tori and the hyper-cube. The switch degree
is 4 for the tori and 16 for the hyper-cube.

\subsection{Task 6.c}
% What can you say about the relative merits of the two topologies?




\appendix
\bibliographystyle{abbrv}
\bibliography{citations}
%%\input{áppendix1}



\end{document}
